\documentclass[10pt]{article}

\usepackage{blindtext} % Package to generate dummy text throughout this template 

\usepackage[sc]{mathpazo} % Use the Palatino font
\usepackage[T1]{fontenc} % Use 8-bit encoding that has 256 glyphs
\linespread{1.05} % Line spacing - Palatino needs more space between lines
\usepackage{microtype} % Slightly tweak font spacing for aesthetics

\usepackage[english]{babel} % Language hyphenation and typographical rules

\usepackage[hmarginratio=1:1,top=32mm,columnsep=20pt]{geometry} % Document margins
%\usepackage[hang, small,labelfont=bf,up,textfont=it,up]{caption} % Custom captions under/above floats in tables or figures
\usepackage[small,labelfont=bf,up,textfont=it,up]{caption}
\usepackage{sidecap}
\usepackage{floatrow}

\usepackage{booktabs} % Horizontal rules in tables

\usepackage{enumitem} % Customized lists
\setlist[itemize]{noitemsep} % Make itemize lists more compact

\usepackage{abstract} % Allows abstract customization
\renewcommand{\abstractnamefont}{\normalfont\bfseries} % Set the "Abstract" text to bold
\renewcommand{\abstracttextfont}{\normalfont\small\itshape} % Set the abstract itself to small italic text

\usepackage{titlesec} % Allows customization of titles
\renewcommand\thesection{\Roman{section}} % Roman numerals for the sections
\renewcommand\thesubsection{\roman{subsection}} % roman numerals for subsections
\titleformat{\section}[block]{\large\scshape\centering}{\thesection.}{1em}{} % Change the look of the section titles
\titleformat{\subsection}[block]{\large}{\thesubsection.}{1em}{} % Change the look of the section titles

\usepackage{fancyhdr} % Headers and footers
\pagestyle{fancy} % All pages have headers and footers
\fancyhead{} % Blank out the default header
\fancyfoot{} % Blank out the default footer
%\fancyhead[C]{CRes User Manual $\bullet$ 2019 $\bullet$ Watson} % Custom header text
\fancyhead[C]{SeismicAirgun User Manual~~~$\bullet$~~~Watson~~~$\bullet$~~~2019} % Custom header text
\fancyfoot[RO,LE]{\thepage} % Custom footer text

\usepackage{titling} % Customizing the title section

\usepackage{hyperref} % For hyperlinks in the PDF

% math packages
\usepackage{amssymb}
\usepackage{amsmath}
\usepackage{amsfonts}

% figure packages
\usepackage{graphicx}

\usepackage{natbib} % bibliography

\usepackage{listings} % for including code snippets
\usepackage{color}

\definecolor{codegreen}{rgb}{0,0.6,0}
\definecolor{codegray}{rgb}{0.5,0.5,0.5}
\definecolor{codepurple}{rgb}{0.58,0,0.82}
\definecolor{backcolour}{rgb}{0.95,0.95,0.92}
 
\lstdefinestyle{mystyle}{
    backgroundcolor=\color{backcolour},   
    commentstyle=\color{codegreen},
    keywordstyle=\color{magenta},
    numberstyle=\tiny\color{codegray},
    stringstyle=\color{codepurple},
    basicstyle=\footnotesize,
    breakatwhitespace=false,         
    breaklines=true,                 
    captionpos=b,                    
    keepspaces=true,                 
    %numbers=left,                    
    numbersep=5pt,                  
    showspaces=false,                
    showstringspaces=false,
    showtabs=false,                  
    tabsize=2
}
 
\lstset{style=mystyle}



%----------------------------------------------------------------------------------------
%	TITLE SECTION
%----------------------------------------------------------------------------------------

\setlength{\droptitle}{-4\baselineskip} % Move the title up

\pretitle{\begin{center}\Huge\bfseries} % Article title formatting
\posttitle{\end{center}} % Article title closing formatting
\title{SeismicAirgun User Guide} % Article title
\author{%
\textsc{Leighton M. Watson} \\% Your name
\normalsize Stanford University \\ % Your institution
\normalsize \href{mailto:leightonwatson@stanford.edu}{leightonwatson@stanford.edu} % Your email address
%\and % Uncomment if 2 authors are required, duplicate these 4 lines if more
%\textsc{Jane Smith}\thanks{Corresponding author} \\[1ex] % Second author's name
%\normalsize University of Utah \\ % Second author's institution
%\normalsize \href{mailto:jane@smith.com}{jane@smith.com} % Second author's email address
}
\date{\today} % Leave empty to omit a date
\renewcommand{\maketitlehookd}{%
}

%----------------------------------------------------------------------------------------

\begin{document}

% Print the title
\maketitle

%----------------------------------------------------------------------------------------
%	ARTICLE CONTENTS
%----------------------------------------------------------------------------------------

\section{Introduction}
{\bf SeismicAirgun} computes airgun/bubble dynamics following a similar treatment to the seminal work of \citet{Ziolkowski1970}. A lumped parameter model is used where the internal properties of the airgun and bubbles are assumed to be spatially uniform. {\bf SeismicAirgun} is written in \texttt{MATLAB} and runs efficiently on a standard desktop/laptop computer. For more details and examples of the application of {\bf SeismicAirgun} see:
\begin{itemize}
\item Chelminski, S., Watson, L. M., and Ronen, S. (2019) Low-frequency pneumatic seismic sources, \emph{Geophysical Prospecting}, \href{https://doi.org/10.1111/1365-2478.12774}{https://doi.org/10.1111/1365-2478.12774}.
\item Watson, L. M., Dunham, E. M., and Ronen, S. (2016) Numerical modeling of seismic airguns and low-pressure sources, \emph{SEG Technical Program Expanded Abstracts}, \href{https://doi.org/10.1190/segam2016-13846118.1}{https://doi.org/10.1190/segam2016-13846118.1}.
\end{itemize}

\section{Directory}
\begin{itemize}
\item {\bf demo} - script files for demonstration
\item {\bf doc} - documentation including user guide and license file
\item{\bf source} - function files associated with numerical implementation
\end{itemize}

{\bf SeismicAirgun} is freely available online at \href{https://github.com/leighton-watson/SeismicAirgun}{https://github.com/leighton-watson/SeismicAirgun} and is distributed under the MIT license (see \texttt{license.txt} for details). 


\newpage
\bibliographystyle{elsarticle-harv}
\bibliography{/Users/lwat054/Documents/Stanford_University/Research/References/Bibliography/library.bib}

\end{document}




